%%%%%%%%%%%%%%%%%%%%%%%%%%%% Document Setup %%%%%%%%%%%%%%%%%%%%%%%%%%%%

% Don't like 10pt? Try 11pt or 12pt
\documentclass[10pt]{article}
\RequirePackage[T1]{fontenc}
\usepackage{graphicx}
\newcommand{\memph}[1]{\textbf{\textit{#1}}}
% LaTeX will typeset using Computer Modern Roman, which a lot of
% non-mathematicians and non-engineers won't like. Also, a few PDF
% viewers may not render CMR very well. Instead, Times New Roman can
% be used. That's what this package does.
\usepackage{times}
\renewcommand{\familydefault}{\sfdefault}
% The automated optical recognition software used to digitize resume
% information works best with fonts that do not have serifs. This
% command uses a sans serif font throughout. Uncomment both lines (or at
% least the second) to restore a Roman font (i.e., a font with serifs).
% (NOTE: This requires the times package above)
%\renewcommand{\familydefault}{\sfdefault}

% This is a helpful package that puts math inside length specifications
\usepackage{calc}

% This package helps LaTeX auto-hyphenate hyphenated words if you use
% special hyphens. For example, bio\-/mimicry will properly hyphenate
% ``mimicry'' if necessary.
\usepackage[shortcuts]{extdash}

% Layout: Puts the section titles on left side of page
\reversemarginpar

%
%         PAPER SIZE, PAGE NUMBER, AND DOCUMENT LAYOUT NOTES:
%
% The next \usepackage line changes the layout for CV style section
% headings as marginal notes. It also sets up the paper size as either
% letter or A4. By default, letter was used. If A4 paper is desired,
% comment out the letterpaper lines and uncomment the a4paper lines.
%
% As you can see, the margin widths and section title widths can be
% easily adjusted.
%
% ALSO: Notice that the includefoot option can be commented OUT in order
% to put the PAGE NUMBER *IN* the bottom margin. This will make the
% effective text area larger.
%
% IF YOU WISH TO REMOVE THE ``of LASTPAGE'' next to each page number,
% see the note about the +LP and -LP lines below. Comment out the +LP
% and uncomment the -LP.
%
% IF YOU WISH TO REMOVE PAGE NUMBERS, be sure that the includefoot line
% is uncommented and ALSO uncomment the \pagestyle{empty} a few lines
% below.
%

%% Use these lines for letter-sized paper
\usepackage[paper=letterpaper,
            %includefoot, % Uncomment to put page number above margin
            marginparwidth=1.2in,     % Length of section titles
            marginparsep=.05in,       % Space between titles and text
            margin=0.9in,               % 1 inch margins
            includemp]{geometry}

%% Use these lines for A4-sized paper
%\usepackage[paper=a4paper,
%            %includefoot, % Uncomment to put page number above margin
%            marginparwidth=30.5mm,    % Length of section titles
%            marginparsep=1.5mm,       % Space between titles and text
%            margin=25mm,              % 25mm margins
%            includemp]{geometry}

%% More layout: Get rid of indenting throughout entire document
\setlength{\parindent}{0in}

% Provides special list environments and macros to create new ones
\usepackage[shortlabels]{enumitem}

% Simpler bibsections for CV sections
% (thanks to natbib for inspiration)
%
% * For lists of references with hanging indents and no numbers:
%
%   \begin{bibsection}
%       \item ...
%   \end{bibsection}
%
% * For numbered lists of references (with hanging indents):
%
%   \begin{bibenum}
%       \item ...
%   \end{bibenum}
%
%   Note that bibenum numbers continuously throughout. To reset the
%   counter, use
%
%   \restartlist{bibenum}
%
%   at the place where you want the numbering to reset.

\makeatletter
\newlength{\bibhang}
\setlength{\bibhang}{1em}
\newlength{\bibsep}
 {\@listi \global\bibsep\itemsep \global\advance\bibsep by\parsep}
\newlist{bibsection}{itemize}{3}
\setlist[bibsection]{label=,leftmargin=\bibhang,%
        itemindent=-\bibhang,
        itemsep=\bibsep,parsep=\z@,partopsep=0pt,
        topsep=0pt}
\newlist{bibenum}{enumerate}{3}
\setlist[bibenum]{label=[\arabic*],resume,leftmargin={\bibhang+\widthof{[999]}},%
        itemindent=-\bibhang,
        itemsep=\bibsep,parsep=\z@,partopsep=0pt,
        topsep=0pt}
\let\oldendbibenum\endbibenum
\def\endbibenum{\oldendbibenum\vspace{-.6\baselineskip}}
\let\oldendbibsection\endbibsection
\def\endbibsection{\oldendbibsection\vspace{-.6\baselineskip}}
\makeatother

%% Reference the last page in the page number
%
% NOTE: comment the +LP line and uncomment the -LP line to have page
%       numbers without the ``of ##'' last page reference)
%
% NOTE: uncomment the \pagestyle{empty} line to get rid of all page
%       numbers (make sure includefoot is commented out above)
%
\usepackage{fancyhdr,lastpage}
\pagestyle{fancy}
\pagestyle{empty}      % Uncomment this to get rid of page numbers
\fancyhf{}\renewcommand{\headrulewidth}{0pt}
\fancyfootoffset{\marginparsep+\marginparwidth}
\newlength{\footpageshift}
\setlength{\footpageshift}
          {0.5\textwidth+0.5\marginparsep+0.5\marginparwidth-2in}
\lfoot{\hspace{\footpageshift}%
       \parbox{4in}{\, \hfill %
                    \arabic{page} of \protect\pageref*{LastPage} % +LP
%                    \arabic{page}                               % -LP
                    \hfill \,}}

% Finally, give us PDF bookmarks
\usepackage{color,hyperref}
\definecolor{darkblue}{rgb}{0.0,0.0,0.3}
\hypersetup{colorlinks,breaklinks,
            linkcolor=darkblue,urlcolor=darkblue,
            anchorcolor=darkblue,citecolor=darkblue}

%%%%%%%%%%%%%%%%%%%%%%%% End Document Setup %%%%%%%%%%%%%%%%%%%%%%%%%%%%


%%%%%%%%%%%%%%%%%%%%%%%%%%% Helper Commands %%%%%%%%%%%%%%%%%%%%%%%%%%%%

%%% HEADING AT TOP OF CURRICULUM VITAE

% The title (name) with a horizontal rule under it
% (optional argument typesets an object right-justified across from name
%  as well)
%
% Usage: \makeheading{name}
%        OR
%        \makeheading[right_object]{name}
%
% Place at top of document. It should be the first thing.
% If ``right_object'' is provided in the square-braced optional
% argument, it will be right justified on the same line as ``name'' at
% the top of the CV. For example:
%
%       \makeheading[\emph{Curriculum vitae}]{Your Name}
%
% will put an emphasized ``Curriculum vitae'' at the top of the document
% as a title. Likewise, a picture could be included:
%
%   \makeheading[{\includegraphics[height=1.5in]{my_picture}}]{Your Name}
%
% the picture will be flush right across from the name. For this example
% to work, make sure the extra set of curly braces is included. Also
% makes ure that \usepackage{graphicx} is somewhere in the preamble.
\newcommand{\makeheading}[2][]%
        {\hspace*{-\marginparsep minus \marginparwidth}%
         \begin{minipage}[t]{\textwidth+\marginparwidth+\marginparsep}%
             {\large \bfseries #2 \hfill #1}\\[-0.15\baselineskip]%
                 \rule{\columnwidth}{1pt}%
         \end{minipage}}

%%% SECTION HEADINGS

% The section headings. Flush left in small caps down pseudo-margin.
%
% Usage: \section{section name}
\renewcommand{\section}[1]{\pagebreak[3]%
    \vspace{1.3\baselineskip}%
    \phantomsection\addcontentsline{toc}{section}{#1}%
    \noindent\llap{\scshape\smash{\parbox[t]{\marginparwidth}{\hyphenpenalty=10000\raggedright #1}}}%
    \vspace{-\baselineskip}\par}

%%% LISTS

% This macro alters a list by removing some of the space that follows the list
% (is used by lists below)
\newcommand*\fixendlist[1]{%
    \expandafter\let\csname preFixEndListend#1\expandafter\endcsname\csname end#1\endcsname
    \expandafter\def\csname end#1\endcsname{\csname preFixEndListend#1\endcsname\vspace{-0.6\baselineskip}}}

% These macros help ensure that items in outer-type lists do not get
% separated from the next line by a page break
% (they are used by lists below)
\let\originalItem\item
\newcommand*\fixouterlist[1]{%
    \expandafter\let\csname preFixOuterList#1\expandafter\endcsname\csname #1\endcsname
    \expandafter\def\csname #1\endcsname{\let\oldItem\item\def\item{\pagebreak[2]\oldItem}\csname preFixOuterList#1\endcsname}
    \expandafter\let\csname preFixOuterListend#1\expandafter\endcsname\csname end#1\endcsname
    \expandafter\def\csname end#1\endcsname{\let\item\oldItem\csname preFixOuterListend#1\endcsname}}
\newcommand*\fixinnerlist[1]{%
    \expandafter\let\csname preFixInnerList#1\expandafter\endcsname\csname #1\endcsname
    \expandafter\def\csname #1\endcsname{\let\oldItem\item\let\item\originalItem\csname preFixInnerList#1\endcsname}
    \expandafter\let\csname preFixInnerListend#1\expandafter\endcsname\csname end#1\endcsname
    \expandafter\def\csname end#1\endcsname{\csname preFixInnerListend#1\endcsname\let\item\oldItem}}

% An itemize-style list with lots of space between items
%
% Usage:
%   \begin{outerlist}
%       \item ...    % (or \item[] for no bullet)
%   \end{outerlist}
\newlist{outerlist}{itemize}{3}
    \setlist[outerlist]{label=\enskip\textbullet,leftmargin=*}
    \fixendlist{outerlist}
    \fixouterlist{outerlist}

% An environment IDENTICAL to outerlist that has better pre-list spacing
% when used as the first thing in a \section
%
% Usage:
%   \begin{lonelist}
%       \item ...    % (or \item[] for no bullet)
%   \end{lonelist}
\newlist{lonelist}{itemize}{3}
    \setlist[lonelist]{label=\enskip\textbullet,leftmargin=*,partopsep=0pt,topsep=0pt}
    \fixendlist{lonelist}
    \fixouterlist{lonelist}

% An itemize-style list with little space between items
%
% Usage:
%   \begin{innerlist}
%       \item ...    % (or \item[] for no bullet)
%   \end{innerlist}
\newlist{innerlist}{itemize}{3}
    \setlist[innerlist]{label=\enskip\textbullet,leftmargin=*,parsep=0pt,itemsep=0pt,topsep=0pt,partopsep=0pt}
    \fixinnerlist{innerlist}

% An environment IDENTICAL to innerlist that has better pre-list spacing
% when used as the first thing in a \section
%
% Usage:
%   \begin{loneinnerlist}
%       \item ...    % (or \item[] for no bullet)
%   \end{loneinnerlist}
\newlist{loneinnerlist}{itemize}{3}
    \setlist[loneinnerlist]{label=\enskip\textbullet,leftmargin=*,parsep=0pt,itemsep=0pt,topsep=0pt,partopsep=0pt}
    \fixendlist{loneinnerlist}
    \fixinnerlist{loneinnerlist}

%%% EXTRA SPACE

% To add some paragraph space between lines.
% This also tells LaTeX to preferably break a page on one of these gaps
% if there is a needed pagebreak nearby.
\newcommand{\blankline}{\quad\pagebreak[3]}
\newcommand{\halfblankline}{\quad\vspace{-0.5\baselineskip}\pagebreak[3]}

%%% FORMATTING MACROS

% Provides a linked \doi{#1} that links doi:#1 to http://dx.doi.org/#1
\usepackage{doi}
% To change the text before the DOI, adjust this command
%\renewcommand\doitext{doi:}

% Provides a linked \url{#1} that doesn't require escape characters
\usepackage{url}

% You can adjust the style \url{} uses here:
% (options are: same, rm, sf, tt; defaults to tt)
\urlstyle{same}

% For \email{ADDRESS}, links ADDRESS to the url mailto:ADDRESS
% (uncomment to typeset the e\-/mail address in typewriter font;
%  otherwise, will be typeset in the \urlstyle above)
%\DeclareUrlCommand\emaillink{\urlstyle{tt}}
\providecommand*\emaillink[1]{\nolinkurl{#1}}
\providecommand*\email[1]{\href{mailto:#1}{\emaillink{#1}}}

\providecommand\BibTeX{{B\kern-.05em{\sc i\kern-.025em b}\kern-.08em \TeX}}
\providecommand\Matlab{\textsc{Matlab}}

% Custom hyphenation rules for words that LaTeX has trouble with
\hyphenation{bio-mim-ic-ry bio-in-spi-ra-tion re-us-a-ble pro-vid-er Media-Wiki}
\long\def\comment#1{}
%%%%%%%%%%%%%%%%%%%%%%%% End Helper Commands %%%%%%%%%%%%%%%%%%%%%%%%%%%

%%%%%%%%%%%%%%%%%%%%%%%%% Begin CV Document %%%%%%%%%%%%%%%%%%%%%%%%%%%%

\begin{document}
\makeheading[\emph{}]{\scalebox{1.5}{Yucheng Zhang }}

\section{Contact Information}

% NOTE: Mind where the & separators and \\ breaks are in the following
%       table. Table is one row made up of three parboxes. The left
%       parbox has address info, the middle parbox has a vertical bar,
%       and the right parbox has phone and electronic contact
%       information.
%
% MACROS: \rcollength is the width of the right column of the table
%             (adjust it to your liking; default is 1.85in).
%         \spacewidth is width of area between left and right boxes.
%
\newlength{\rcollength}\setlength{\rcollength}{1.85in}%
\newlength{\spacewidth}\setlength{\spacewidth}{20pt}
%
\begin{tabular}[t]{@{}p{\textwidth-\rcollength-\spacewidth}@{}p{\spacewidth}@{}p{\rcollength}}%

% Address box
\parbox{\textwidth-\rcollength-\spacewidth}{%
229-3755 W 6th Ave\\
Vancouver, BC V6R1T9, Canada
}

&
% Uncomment to add a vertical bar in middle of contact information
%{\vrule width 0.5pt}
\parbox[m][\baselineskip]{\spacewidth}{} &

% Non-snail-mail contact information
\parbox{\rcollength}{%
\emph{Mobile:}~778-322-3307 \\
\emph{Email:}~\email{yuchengz@ece.ubc.ca}\\
%\emph{WWW:} \href{http://www.tedpavlic.com/}{www.tedpavlic.com}
}

\end{tabular}

%%
%% In modern CV's, it seems like ``Objective'' is frowned upon. Instead,
%% incorporate it into a well-constructed cover letter. The ``More
%% information'' can go at the end of the CV, but it should not distract
%% from the section giving references available to contact.
%%
%
% \section{Objective}
%
% Full-time position that allows for advanced research in electrical and
% computer engineering (communications, control, software, electronics,
% and sustainability), with a particular focus on complex distributed
% systems (i.e., modeling, analysis, design, and verification)
% \begin{innerlist}
%     \item For more information, see \url{http://www.tedpavlic.com/engjobsearch/}
% \end{innerlist}

%\section{Goal}
%To do exciting and influential research

%I am interested in tools and techniques for reliable software.
%My current research is mainly focused on 
%I find it very exciting to explore new approaches to ensure \textbf{security} in software. 
%I also have some research experience in \emph{cloud computing}, \emph{embedded system} and know some about \emph{virtualization}. 
\comment{
\section{Research Interests}
My research interests are \textbf{security}, \textbf{distributed system} and \textbf{software engineering}.
}

\section{Education}

\textbf{The University of British Columbia}~(UBC), Vancouver, British Columbia, Canada
\begin{outerlist}
\item[] MASc. in {Computer Engineering}%, Department of Computer Science
             \hfill 9/2014 -- Present
        \begin{innerlist}
        \item Advisor:
              Dr. \href{https://www.ece.ubc.ca/~amesbah/}
                   {Ali Mesbah}
         \item GPA: 4.0/4.33
        \end{innerlist}

\end{outerlist}

\halfblankline

\textbf{The University of British Columbia}~(UBC), Vancouver, British Columbia, Canada
\begin{outerlist}
\item[] B.Sc. in {Computer Science} and {Statistics}%, Department of Computer Science
             \hfill 9/2009 -- 5/2014
        \begin{innerlist}
         \item GPA: 3.85/4.33
        \end{innerlist}

\end{outerlist}

\halfblankline

\textbf{University of Waterloo}, Ontario, Canada
\begin{outerlist}
\item[] Mathematics/Business Honors (Transfer Credit)
             \hfill 9/2008 -- 12/2008

\end{outerlist}


\section{Awards and Honors}
\begin{innerlist}
%8 in dept
\item ACM SIGSOFT Grant 		\hfill	2015
\item Graduate student travel fund of University of British Columbia \hfill 2015
\item University of British Columbia President's Entrance Scholarship  \hfill 2009
\item University of Waterloo President's Scholarship \hfill 2008
\end{innerlist}

\section{Research Experience}
\textbf{Research Assistant},The University of British Columbia, Vancouver, British Columbia, Canada \hfill Sep 2014 -- Present
\begin{outerlist}
\item[] Advisor: Dr. Ali Mesbah
\item[] \textbf{Assertions Are Strongly correlated With Test Suite Effectiveness}
\begin{innerlist}
\item Designed and conducted controlled experiments to verify the influence of assertions on test suite effectiveness from three different angles: assertion quantity, assertion coverage, and assertion types.
\item Built an infrastructure to automatically compose test suites of target properties and executed through mutation testing.
\item Statistically analyzed the experiment results by using correlations, significant tests, and analysis of variance.
\item Published one research paper at \textbf{ESEC/FSE 2015}.
\end{innerlist}

\item[]\textbf{Why Assertion Types Influence Test Suite Effectiveness}
\begin{innerlist}
\item Customized PIT (a very popular mutation testing tool) to collect mapping information between test assertions and detectable mutants.
\item Built an infrastructure to automatically collect assertion related properties, such as assertion content types,  direct/indirect assertion coverage, assertion density, cyclomatic complexity, and etc.
\item Study and make understanding of the experiment result from different angles, ex: relationship between assertions content types and mutation operators
\end{innerlist}

\end{outerlist}

\section{Publications}

\begin{bibenum}

    \item \textbf{Assertions Are Strongly Correlated With Test Suite Effectiveness}\\
    \textbf{Yucheng Zhang} and Ali Mesbah \\
 \textit{In Proc. the 10th Joint Meeting of the European Software Engineering Conference and the ACM SIGSOFT Symposium on the Foundations of Software Engineering (\textbf{ESEC/FSE 2015})}. Bergamo, Italy. (Acceptance Rate: 25.4\%)
\end{bibenum}

\section{Academic Activities}
\textbf{Conference Talks}
\begin{innerlist}
\item	Assertions Are Strongly Correlated With Test Suite Effectiveness 
\newline ACM SIGSOFT Symposium on the Foundations of Software Engineering (FSE 2015)
\end{innerlist}

\section{Teaching Experience}
\textbf{Teaching Assistant}, The University of British Columbia
\begin{innerlist}
\item Economic Analysis of Engineering Projects (CPEN 481 ) \hfill	1/2016 -- Present
\item Software Testing (CPEN 422)	\hfill	9/2015 -- 12/2015
\item Introduction to Probability (Math 302)	\hfill	1/2014 -- 5/2014
\item Applied Linear Algebra (Math 307)	\hfill	9/2013 -- 12/2013
\item Differential Calculus with Physical Applications (Math 180)	\hfill	9/2013 -- 12/2013
\end{innerlist}

\section{Working Experience}
\textbf{Samsung Research America}~(SRA), Mountain View, CA, USA \hfill 5/2015 -- 9/2015
\begin{outerlist}
\item[] Software Engineer Intern
\begin{innerlist}
\item Supervisor: Bruce Hoenig, David Paquet
\item Designed and maintained a database for the GPU research team based on complex design requirements.
\item Analyzed and visualized data generated by various processes of the Advanced Processor Lab to provide insights on user volume consumption, job delays, etc.
\item Implemented a team profile management tool which are now being used by the entire Advanced Processor Lab.
\end{innerlist}
\end{outerlist}

\halfblankline

\textbf{NEXSM Inc.},Vancouver, British Columbia, Canada \hfill	6/2014 -- 7/2014
\begin{outerlist}
\item[] Database Management and Web Developer
\begin{innerlist}
\item Independently designed and implemented a social media application.
\end{innerlist}
\end{outerlist}

\halfblankline

\textbf{School of Music, The University of British Columbia}, Canada \hfill 	5/2015 -- 8/2013
\begin{outerlist}
\item[] Database Management and Web Developer Assistant
\begin{innerlist}
\item Independently designed and implemented a student information management and practice room hours booking system.
\item Designed and implemented database of the application. 
\item Configured and deployed the application on a Redhat Linux server.
\item The application is currently being used by school of music and their students.
\end{innerlist}
\end{outerlist}

\halfblankline

\textbf{Gumstix Research (Canada) Ltd.}, Vancouver, BC, Canada \hfill 	1/2012 -- 8/2012
\begin{outerlist}
\item[] Software Engineer
\begin{innerlist}
\item Refactored the webERP company management system and osCommerce online store using Symfony2 as framework, Doctrine as database management tool, and Twig as template.
\item Independently developed a product selector tool using JavaScript and DOJO to make it easier for customers to understand and select their desired products.
\item Improved and added new features to the existing web application as required by customers.
\item Assisted system admin in implementing auto application installer using Python.
\end{innerlist}
\end{outerlist}

\section{Technical Skills}
\begin{innerlist}
\item \textit{Programming Languages}:~Java, C++, C, PHP, JavaScript, Python, HTML, SQL, CSS
%~Good command of Java, JavaScript, C\#, Python, Flex. Familiar with Scala, C/C++, HTML. Experience with Assembly
\item \textit{Develop Environments}:~Eclipse, Visual Studio, IntelliJ, gcc

\item {Specialty}:~Algorithm Design, Software Testing, Full-Stack Web Development, Data Analysis and Modeling
\end{innerlist}

\section{Activities and Interests}
\begin{innerlist}
\item Accepted professional violin training for 10+ years, and achieved Level 9 (\textbf{the highest technical level} issued by Chinese Musicians' Association) at the age of 15
\item First prize of provincial compitetion of Go
\item Won the Euclid Mathematics Contest at Waterloo in 2008

\item Making handcrafts including beading, knitting and crocheting


\end{innerlist}

%\section{More Information}
%More information and auxiliary documents can be found at\\%
%\url{http://stap.sjtu.edu.cn/index.php?title=Qi_Wang}.


\end{document}

%%%%%%%%%%%%%%%%%%%%%%%%%% End CV Document %%%%%%%%%%%%%%%%%%%%%%%%%%%%%
